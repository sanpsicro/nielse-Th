% ----------------------------------------------------------------
% AMS-LaTeX Paper ************************************************
% **** -----------------------------------------------------------
\documentclass{amsart}
\usepackage{graphicx}
\usepackage{color}
\usepackage{amsmath}
\usepackage[psamsfonts]{amssymb}
% ----------------------------------------------------------------
\vfuzz2pt % Don't report over-full v-boxes if over-edge is small
\hfuzz2pt % Don't report over-full h-boxes if over-edge is small
% THEOREMS -------------------------------------------------------
\newtheorem{thm}{Theorem}[section]
\newtheorem{cor}[thm]{Corollary}
\newtheorem{lem}[thm]{Lemma}
\newtheorem{prop}[thm]{Proposition}
\theoremstyle{definition}
\newtheorem{defn}[thm]{Definition}
\theoremstyle{remark}
\newtheorem{rem}[thm]{Remark}
\numberwithin{equation}{section}
% MATH -----------------------------------------------------------
\newcommand{\norm}[1]{\left\Vert#1\right\Vert}
\newcommand{\abs}[1]{\left\vert#1\right\vert}
\newcommand{\set}[1]{\left\{#1\right\}}
\newcommand{\Real}{\mathbb R}
\newcommand{\eps}{\varepsilon}
\newcommand{\To}{\longrightarrow}
\newcommand{\BX}{\mathbf{B}(X)}
\newcommand{\A}{\mathcal{A}}
% ----------------------------------------------------------------
\begin{document}

\AmS -\LaTeX

\title[Nielsen]{Nielsen}%
\author{san}%
%\address{}%
%\email{}%

%\thanks{}%
%\subjclass{}%
%\keywords{}%

%\date{}%
%\dedicatory{}%
%\commby{}%
% ----------------------------------------------------------------
\begin{abstract}
a
\end{abstract}
\maketitle
% ----------------------------------------------------------------
\section{0}

Grupos propiamente discontinuos de transformaciones fraccionales
lineales con c\'irculo principal en el sentido de la teor\'ia de
automorfismos. \\

transformaciones en el plano hiperb\'olico, caracterizaci\'on de
grupos discontinuos de los anteriores. \\

Las lineas rectas y \'angulos se trataran como en el sentido de
poincar\'e. \\

Todas las cosecuencias geom\'etricas se entenderan en el sentido
no-euclidio. \\

Utilizaremos el modelo del disco de Poincar\'e $S^{1}$ cuyo interior
le denominaremos $D$ y cuya frontera le denominaremos  $E$.

Se llamar\'an puntos finales a los extremos de una recta situados en
el c\'irculo infinito. \\

Identificaremos tres tipos de relaciones entre las rectas, a saber,
concurrentes, paralelas y divergentes. \\

Las rectas \texttt{paralelas} son dos rectas que tienen un punto
final en com\'un.\\

Las rectas \texttt{concurrentes } son las que se intersectan en $D$.
\\

Las rectas \texttt{divergentes} son las que no se intersectan en
nigun lugar. \\

\textcolor[rgb]{1.00,0.00,0.00}{Dibujos} \\

Los \texttt{orociclos} son circulos tangeentes a alg\'un punto en la
frontera $E$ de $S^{1}$. Estos no son geod\'esicas. \\


Los \texttt{hiperciclos} se forman dada una recta $\emph{L}$ y un
punto $\textbf{P}$ fuera de ella se considera el lugar geom\'etrico
de todos los puntos a distancia $d_{H}(\textbf{P},\emph{L})$ de
$\emph{L}$ y del mismo lado de $P$
(\textcolor[rgb]{1.00,0.00,0.00}{Cualquier geod\'esica determina dos
lados}). De igual manera los hyperciclos no son geod\'esicas. \\

\textbf{Transformaciones isom\'etricas} \\

Dividiremos las transformaciones isom\'etricas en dos grupos, los
que preservan la orientaci\'on y los que no. En el grupo de los que
preservan la orientaci\'ion distinguimos tres tipos distintos de
transformaciones; \\

Las transformaciones \texttt{parab\'olicas} tienen un punto fijo en
$E$ y dado un orociclo $K$ tangente en dicho punto fijo a $E$, esta
transformaci\'on tiene un desplazamiento fijo en $K$.
\textcolor[rgb]{1.00,0.00,0.00}{Dibujo} \\

Las tranformaciones \texttt{el\'ipticas} tienen un punto fijo en $D$
y un \'angulo fijo $\phi$ de rotaci\'on en torno al punto fijo.
\textcolor[rgb]{1.00,0.00,0.00}{Dibujo} \\

Las transformaciones \texttt{hiperb\'olicas} tienen dos puntos fijos
en $E$ y una recta invariante, la recta que pasa por dichos puntos
fijos a la cual llamaremos eje de la transformaci\'on.
\textcolor[rgb]{1.00,0.00,0.00}{Dibujo} \\

$G$ es un grupo de transformaciones isom\'etricas, un punto
invariante para $G$ si es fijo bajo todos los elementos de $G$, de
una manera similar pensamos en una recta invariante. \\

Un par de conjuntos $M$ y $M'$ en $D +E$ son equivalentes respecto a
$G$ si existe $f \in G$ tal que $M = fM'$. El total de conjuntos
equivalentes de $M$ se denota como la familia $GM$ llamada clase de
equivalencia de $M$, en esta familia si $f_{1},f_{2} \in G  $ y
$f_{1} \neq f_{2 }$ entonces $f_{1}M \neq f_{2}M$ sin importar que
como conjunto sean iguales. \\

Decimos que $G$ es discontinuo en un punto $x \in D$,si $x$ no es un
punto de acumulaci\'on de la clase $Gx$, si esto sucede $Gx$ no
tiene puntos de acumulaci\'on en $D$
(\textcolor[rgb]{1.00,0.00,0.00}{Prueba}), esto implica que si $G$
es discontinuo en un punto de  $D$ entonces es discontinuo en todo
$D$. (\textcolor[rgb]{1.00,0.00,0.00}{prueba}). \\

Las \'unicas isometr\'ias de $H^{+} $ que preservan la orientaci\'on
son las de $PSL(2,\mathbb{R})$, las transformaciones mencionadas con
anterioridad se encuentran en este grupo. \\

Si $f \in PSL(2,\mathbb{R})$ y $g$ alguna otra transformaci\'on no
necesariamente en $PSL(2, \mathbb{R})$ entonces $gfg^{-1}$ es del
mismo tipo que $f$.\\ \\

Llamaremos \texttt{reversiones} a las transformaciones que invierten
la orientaci\'on, las reversiones son las \texttt{reflexiones}
respecto a una recta, y las reflexiones compuestas con elementos
hiperbolicos a las que llamaremos \texttt{h-reflexion}. \\ \\


La reflexi\'on respecto a una recta hiperb\'olica es la reflexi\'on
respecto a una circunferencia euclidiana. \\ \\


\textbf{Teorema principal}

\begin{thm}
Una condici\'on necesaria y suficiente para que un grupo de
transformaciones isom\'etricas en el plano hiperb\'olico sin puntos
ni lineas invariantes sea discontinuo en $D $ es que los puntos
fijos de los elementos elipticos del grupo, si los hay, no se
acumulen en $D$

\end{thm}

La condici\'on de que no tenga puntos ni rectas invariantes es
necesaria. \\ \\


\textcolor[rgb]{1.00,0.00,0.00}{contraejemplo} \\

Si $G$ es un grupo de elementos parab\'olicos con un mismo punto
fijo al infinito.\\ \\

Si $G$ es un grupo de elementos hiperb\'olicos con una recta fija
dada. \\ \\

ninguno de los anteriores contiene elementos elipticos sin embargo
ninguno es discontinuo en $D$.
(\textcolor[rgb]{1.00,0.00,0.00}{prueba?}) \\ \\

Si $G$ contiene reversiones  y $H \subseteq G$ y adem\'as $H
\subseteq PSL(2,\mathbb{R})$ se tiene que $H$ es subgrupo de $G$ y
adem\'as es normal y de \'indice 2, es normal ya que si $f\ in H$
para cualquier elemento $g \in G$ se tiene que $gfg^{-1}$ es del
mismo tipo que $f$ por lo que $gfg^{1} \in H$, es de \'indice 2 y
para ver esto basta probar que si $f,g \in G - H$ entonces $f \cong
g$ equivalentemente que $fg^{-1} \in H$, dado que $f , g$ revierten
orientaci\'on  se tiene que $fg^{-1}$ conserva la orientaci\'on y
por lo tanto esta en
$H$. \\ \\

Una clase de equivalengia $Gx$ es la suma de las clases $Hx$ y $Hgx$
con $g$ una reversi\'on en $G$ ($H \cup Hg = G \Rightarrow Hx \cup
Hgx = Gx $ ), por lo que se concluye que $G$ y $H$
son ambos discontinuos en $D$ o ambos no lo son. \\ \\

En otras palabras $H$ es discontinuo en $D$ si y solo si $G$ lo es.
\\ \\
\textbf{ Si $G$ no tiene rectas ni puntos invariantes tampoco los
tiene $H$.} Supongamos que $H$ deja invariantes un punto en $D +E$,
pero ning\'un otro punto (y por tanto ninguna recta), como $G$ no
deja puntos invariantes, existe $g \in G $ tal que $gc \neq c$ luego
el subgrupo $gHg^{-1}$ dea invariante $gc$ pero ningun otro punto en
$D+E$ pero como $H \vartriangleleft G$ se tiene que  $gHg^{-1}$ y
como $gc \neq c $ implica que $H$ fija $c$ y $gc$ lo cual es una
contradicci\'on, concluimos que $H$ no deja puntos invariantes. Un
razonamiento analogo nos ense�a que $H$ no deja rectas invariantes.
\\ \\



Desde ahora  $G$ denotar\'a un subgrupo de $PSL(2,\mathbb{R})$ sin
puntos ni rectas invariantes. \\ \\

\begin{thm}[caso 1] Cuando G tiene elementos elipticos \textcolor[rgb]{1.00,0.00,0.00}{-Enunciado teorema-}
\end{thm}

\emph{Prueba;} \\ \\

Primero probaremos que si los centros de los elementos elipticos se
acumulan $\Rightarrow$ $G$ no es discontinuo en $D$ \\ \\

Supongamos que los puntos fijos de los elementos elipticos se
acumulan en $D$ (tiene un punto de acumulaci\'on en $D$), es posible
encontrar una sucesi\'on de puntos fijos que convergan al punto de
acumulaci\'on digamos $x$(\textcolor[rgb]{1.00,0.00,0.00}{por ser
espacio m\'etrico})y sean estos $y_{1},y_{2},...$ y
$f_{1},f_{2},...$ los respectivos elementos el\'ipticos. Por un lado
$d_{H}(y_{i},x) \leq \eps $ para $i$ adecuada, considerando $\lbrace
f_{i}x\rbrace$ podemos notar que

$$ d_{H}(f_{i}x,x) \leq d_{H}(f_{i}x,y_{i}) + d_{H}(y_{i},x) = 2d_{H}(y_{i},x)$$

lo anterior debido a que $f_{i}$ fija $y_{i}$ y que es una
isometr\'ia. Luego entonces $f_{i}x \rightarrow x$, es decir $Gx$ se
acumula en $x$  por tanto $G$ no es discontinuo en $x$ y por lo
tanto no lo es en $D$.\\ \\


Si los puntos fijos de los elementos el\'ipticos no se acumulan
$\Rightarrow$ $G$ es discontinuo en $D$.\\ \\

elegimos un elemento el\'iptico $f $ y sea $c$ su punto fijo, como
$G$ no tiene puntos invariantes existe $g \in G $ tal que $gc \neq
c$ y adem\'as $gc$ es punto fijo de $gfg^{-1} $ que es de igual
manera un elemento el\'iptico. Sea $G(c)$ el subgrupo de $G$ de
elementos el\'ipticos con punto fijo $c$ entonces $\sharp G(c) <
\infty$, para ver esto notemos 2 cosas;

\begin{enumerate}
\item $hgc$ es punto fijo de alg\'un elemento el\'iptico en $G$ a
saber $hgfg^{-1}h^{-1}$
\item $d_{H}(gc,c)=d_{H}(hgc,hc)=d_{H}(hgc,c)$
\end{enumerate}

esto implica que el conjunto $G(c)gc$ esta contenido en una
circunferencia con centro $c$, si $G(c)gc$ es infinito y dado que la
circunferencia es compacta $G(c)gc$ tendria un punto de
acumulaci\'on en dicha circunferencia lo cual contradice nuestra
hip\'otesis, entonces necesariamente  $\sharp G(c) < \infty$, luego
se tiene que $Gc$ se compone de puntos fijos de elementos
el\'ipticos (gc es punto fijo de $gfg^{-1}$ ) en $G$. \\

Si $gc \in Gc$ se tiene  $\sharp G(c) = \sharp G(gc)$ ya que se
tiene la biyecci\'on ... (\textcolor[rgb]{1.00,0.00,0.00}{mejorar
prueba})



\section{1}

\begin{lem}[lema 1]
Sean f y g elementos hiperb\'olicos cuyos ejes son divergentes y
cuyos desplazamientos $\lambda$ miden lo mismo pero tienen sentidos
opuestos, y sea $\delta$ la distancia entre sus ejes, entonces $fg$
es el\'iptico, parab\'olico o hiperb\'olico dependiendo de
$$senh \frac{1}{2} \delta senh \frac{1}{2} \lambda    \left \{ \begin{matrix} > 1 & \mbox{ }\mbox{ }
\\ =1 & \mbox{ }\mbox{ } \\ <1 \mbox{} {}\end{matrix}\right. $$
\end{lem}

\emph{Prueba.} \\ \\

Como los ejes son divergentes existe una normal com\'un a ambos y a
esta recta le llamamos $s'$, denotemos a los ejes de $f$ y $g$ como
$A_{f}$ y $A_{g}$ respectivamente, entonces la normal que
mencionamos antes corta a $A_{f}$ en $a$ y a $A_{g}$ en $a'$ y
elegimos $b,b'$ en $A_{f}$ y $A_{g} $ respectivamente  de tal manera
que $d_{H}(a,b)=d_{H}(a',b')= \frac{1}{2} \lambda$ y adem\'as en la
direcci\'on que se desplaza $f$. Sean $s,s''$ las rectas normales a
$A_{f},A_{g}$ en $b, b'$ y por abuso de notaci\'on las reflexiones
respecto a estas rectas y $s'$ se denotaran de la misma manera. \\ \\

Queremos ver que tenemos la relaci\'on $f=ss'$, es claro que $ss'$
preserva $A_{f}$ ya que $s'$ lo preserva y $s$ tambi\'en por ser
normal a este eje (\textcolor[rgb]{1.00,0.00,0.00}{las
circunferencias ortogonales son invariantes bajo inversiones
respecto a ellas}), para ver que tiene el mismo sentido que $f$
basta ver hacia donde desplaza un punto y es facil tomando a que
queda fijo bajo $s'$ y bajo $s$ se desplaza en la misma direcci\'on
que $f$, por \'ultimo hay que verificar que $d_{H}(x,ss'x) = \lambda
$ pero como $ss'$ preserva la orientaci\'on y tiene dos puntos fijos
(los puntos finales del eje fijo) necesariamente tiene que ser un
alemento hiperb\'olico y para saber cuanto mide su desplazamiento de
nuevo basta verificarlo para un punto en la recta fija y tomando a
de nuevo $a$ notamos que $d_{H}(a,ss'a)= d_{H}(a,sa)= \lambda$, en
conclusi\'on $f=ss'$ y un proceso analogo nos permite probar que
$g=s's''$ (\textcolor[rgb]{1.00,0.00,0.00}{dibujos}). \\ \\

Queremos saber ahora que tipo de transformaci\'on es $fg$, por lo
anterior tenemos que $fg= ss's's'' = ss''$, ahora todo depende de la
posici\'on de los ejes de $f$ y $g$. \\ \\


\textbf{ Caso 1.}[las rectas se intersectan el alg\'un punto de
$D+E$](\textcolor[rgb]{1.00,0.00,0.00}{dibujo}) \\ \\


Sea $m$ el punto medio del segmento $aa'$, la normal a $s' $ que
pasa por $m$ bisecta el \'angulo que forman los ejes en el punto de
intersecci\'on $p$ (\textcolor[rgb]{1.00,0.00,0.00}{dibujo}), para
ver esto reflejamos las rectas $l$ y $\sigma$ respecto de $l$
obtenemos $l$ y $s''$, como la reflexi\'on es conforme el \'angulo
entre $l$ y $s''$ es el mismo que el \'angulo entre $l$ y $s$. Es
claro que el punto de intersecci\'on $p$ sera el punto fijo de la
transformaci\'on y sea $\frac{1}{2} \phi_{fg}$ el \'angulo entre $s$
y $s'$ (\textcolor[rgb]{1.00,0.00,0.00}{la inversion respecto a dos
circunferencias es una rotacion con angulo el doble del angulo de
esta interseccion}) (\textcolor[rgb]{1.00,0.00,0.00}{dibujo}). \\


Luego $abpm$ es un cuadrilatero con 3 \'angulos rectos y el \'angulo
restante agudo (cuadrilatero de lambert) cuyas medidas son
$\frac{1}{4} \phi_{fg}$ con lados $\frac{1}{2} \delta$ y $
\frac{1}{2} \lambda$ por lo cual tenemos
(\textcolor[rgb]{1.00,0.00,0.00}{dibujo}) $senh \frac{1}{2} \delta
senh \frac{1}{2} \lambda = cos \frac{1}{4} \phi_{fg} \leq 1$ y es
igual a 1 en el caso de que se intersecten en el infinito ( $\phi
=0$)y en dado caso la transformaci\'on seria parab\'olica. \\ \\


\textbf{Caso 2} \\ \\


En el caso que los ejes sean divergentes, existe una recta normal a
ambos  y se tiene que la recta $l$ bisecta al segmento entre $s$ y
$s'$ de esta recta normal, entonces para estar seguros probaremos;

\begin{enumerate}
\item Esta recta normal es el eje de $fg$
\item En efecto $l$ bisecta al segmento entre $s$ y $s'$ de esta
recta normal
\end{enumerate}

Para (1) Probaremos que $fg$ es hiperb\'olica viendo que tiene dos
puntos fijos y que se encuentran en la normal mencionada antes. Sean
$x$ y $x' $ los puntos finales de dicha recta normal y notemos que
$s''x=x'$ ya que $s''x$ debe cumplir que esta sobre la recta normal
a $s''$ y que $d_{H}(x,s'')=d_{H}(s''x,s'')$ pero $d_{H}(x,s'') =
\infty$, por la misma raz\'on $sx'=x$ entonces $ss''x=x$ y tambien
se cumple $ss''x'=x'$ por lo tanto $fg$ tiene 2 puntos fijos y su
eje es la recta que los une, que en este caso es la normal
mencionada. (\textcolor[rgb]{1.00,0.00,0.00}{dibujo}) \\ \\

Para (2) queremos probar que $d_{H}(x,y) = d_{H}(y,x')$ donde $y$ es
el punto de intersecci\'on de la normal con $l$, al reflejar
respecto de $l$ y notando que $x \mapsto x'$ bajo $s$ y $d_{H}(x,y)
= d_{H}(y,sx)$ tenemos el resultado. \\ \\


Finalmente tenemos las rectas $s'',A_{g},s,l$ y $A_{fg}$ y estas
forman un pentagono con 4 \'angulos rectos y con lados
$\frac{1}{2}\lambda_{fg},\frac{1}{2}\delta,\frac{1}{2}\lambda$
(\textcolor[rgb]{1.00,0.00,0.00}{dibujo}) luego tenemos
(\textcolor[rgb]{1.00,0.00,0.00}{apendice})
$$ senh \frac{1}{2} \delta senh \frac{1}{2} \lambda = cosh \frac{1}{4} \lambda_{fg} >1$$
$ _{\square} $ \\ \\


\section{2}

\begin{lem}[lema 2]
Sean $f$ y $g$ dos elementos hiperb\'olicos con ejes $A_{f}$ y
$A_{g}$ y desplazamientos iguales $\lambda $, y con \'angulo de
intersecci\'on $\phi $ entonces el conmutador $[f,g]=fgf^{-1}g^{-1}$
es el\'iptico , parab\'olico o hiperb\'olico dependiendo si

$$sen \phi senh^{2}\frac{1}{2} \lambda  \left \{ \begin{matrix} >1
\\ =1 \\ < 1 \end{matrix}\right. $$
\end{lem}

\emph{Prueba.}\\ \\

Sea $s'$ el punto de intersecci\'on de $A_{f}$  y $A_{g}$, sea $s$
el punto sobre $A_{f}$ tal que $d_{H}(s',s)= \frac{1}{2} \lambda$ y
trazado siguiendo la direcci\'on de $A_{f}$, y sea $s''$ el punto
sobre $A_{g}$ tal que $d_{H}(s',s'')=\frac{1}{2} \lambda$ y trazado
siguiendo la misma direcci\'on que $A_{g}$.(\textcolor[rgb]{1.00,0.00,0.00}{dibujo}) \\ \\

Y tomemos las rotaciones de \'angulo $\pi$  respecto a estos puntos
y las denotaremos igual que denotamos a sus puntos fijos. \\ \\

Queremos primero probar que $f=ss'$, para esto deseamos ver que
puntos fija $ss'$. La rotaci\'on por angulo $\pi$ funciona de la
siguiente manera; dado un punto $x$ se traza la recta que va de $x$
al centro de rotaci\'on $c$ el rotado de $x$ llamemosle $y$ esta
sobre la misma recta y $d_{H}(x,c)=d_{H}(y,c)$, con este proceso
podemos ver que $ss'$ fija los mismos puntos que $f$ por lo tanto el
eje de $f$ es el mismo ee de $ss'$ solo falta ver cuanto es el
desplazamiento de $ss'$ y como antes, basta verificarlo para un
punto y en esta ocasi\'on tomamos el punto $s'$ y tenemos que
$d_{H}(s,ss'(s'))=d_{H}(s,s(s'))= 2d_{H}(s',s)=2(\frac{1}{2}
\lambda)= \lambda $ por lo tanto $f=ss'$. Un proceso analogo nos
ense�a que $g=s's''$ y como $s's'$ es la indentidad tenemos que $fg
= ss''$, m\'as a\'un $f^{-1} = s's,g^{-1}=s''s'$ y entonces
$f^{-1}g^{-1}= s'ss''s'=s'fgs'=s'fgs'^{-1}$, luego $ss'$ es un
elemento hiperb\'olico con recta invariante la recta que pasa por
$s$ y $s''$ y desplazamiento $2d_{H}(s,s'')$, m\'as a\'un
$f^{-1}g^{-1}$ es hiperb\'olico obtenido de $fg$ conjungando por
$s'$ tenemos entonces;

\begin{enumerate}
\item $f^{-1}g^{-1}$ tiene un desplazamiento con la misma medida que
$fg$
\item su eje $A_{f^{-1}g^{-1}}$ se obtiene de $A_{fg}$ rotando por $s'$.
\end{enumerate}

Para probar (1) queremos ver cuanto mide $d_{H}(x,s'fgs'x)$ y
tenemos que $d_{H}(x,s'fgs'x)= d_{H}(s'x,fg(s'x))$ y esta \'ultima
cantidad es el desplazamiento de $fg$. \\ \\

(2) se cumple  ya que $f^{-1}g^{-1} (s'A_{fg})= s'fgs'(s'A_{fg})=
s'fg(A_{fg})=s'A_{fg}$. \\ \\

Entonces estos ejes son divergentes y sus correspondientes
transformaciones tienen sentidos opuestos, aplicamos entonces el
\textbf{lema 1} a $fg$ y a $f^{-1}g^{-1}$ entonces $[f,g]$ es
el\'iptico , parab\'olico o hiperb\'olico seg\'un sea

$$senh \eta senh \frac{1}{2} \lambda_{fg} \lesseqgtr 1$$(\textcolor[rgb]{1.00,0.00,0.00}{dibujo})

Donde $\eta$ es la perpendicular desde $s'$ en el tri\'angulo
$ss's''$ y por trigonometr\'ia hiperb\'olica
(\textcolor[rgb]{1.00,0.00,0.00}{apendice}) se cumple $senh \eta
senh\frac{1}{2} \lambda_{fg} = sen \phi senh^{2} \frac{1}{2}\lambda$
$_{\square}$

\section{3}

\begin{lem}[lema 3]
Sean $f,g$ elementos parab\'olicos con puntos fijos $c_{f}$ y
$c_{g}$ entonces el conmutador $[f,g]$ es hiperb\'olico.
\end{lem}

\emph{Prueba.} \\ \\

(\textcolor[rgb]{1.00,0.00,0.00}{dibujo}) Sea $s'$ la recta que une
$c_{f}$ y $c_{g}$ y denotemos la reflexi\'on respecto a esta recta
tambi\'en como $s'$. Sea $O_{f}$ un oriciclo dado con punto al
infinito $c_{f}$, y $x \in O_{f}$ y sea $x'$ el punto medio de la
recta (\textcolor[rgb]{1.00,0.00,0.00}{falta probar que existen las
rectas cuyas reflexiones nos dan la tras�nsformacion buscada})

Tenemos entonces que $f=ss�$, y analogamente obtenemos  que existe
$s''$ tal que $g=s's''$ luego tenemos
$[f,g]=ss''s'ss''s'=(ss''s')^{2}$, la composici\'on de reflexiones
conserva el sentido por lo tanto la composici\'on de 3 reflexiones
invierte el sentido y es por tanto una reflexi\'on o una
h-reflexi\'on entonces su cuadrado es la identidad o un elemento
hiperbolico, como $f$ y $g$ no permutan
(\textcolor[rgb]{1.00,0.00,0.00}{�porque?}) no puede ser una la
identidad por lo tanto es un elemento hiperb\'olico. $_{\square}$

\section{4}

\begin{lem}[lema 4]
Sean $f$ y $g$ elementos hiperb\'olicos con ejes paralelos y con
punto com\'un $u$, entonces el conmutador $[f,g]$ es un elemento
parab\'olico con centro en $u$
\end{lem}

(\textcolor[rgb]{1.00,0.00,0.00}{dibujo})

\emph{Prueba.} \\ \\

Primero veamos que  $f \neq gf^{-1}g^{-1}$, es claro que
$gf^{-1}g^{-1}$ fija $u$ ya que tanto $f$ como $g$ lo fijan, sea $x$
el punto final del eje de $f$ distinto de $u$, es claro que $g^{-1}$
no fija este punto, y $gx \neq u$ (puesto que $g^{-1}u \neq x$)
concluimos con esto $f \neq gf^{-1}g^{-1}$, claramente su
desplazamiento mide lo mismo (debido a que $g$ es isometria) pero
con sentidos opuestos (ya que $gf^{-1}g^{-1}$ respeta el sentido de
$f^{-1}$). (\textcolor[rgb]{1.00,0.00,0.00}{dibujo})
(\textcolor[rgb]{0.00,0.00,1.00}{inconcluso}).

\section{5}

\begin{lem}[lema 5]
Si en un grupo $G < PSL(2,\mathbb{R})$ sin puntos invariantes, 2
ejes de elementos hiperb\'olicos son paralelos entonces el grupo
contiene un elemento el\'iptico
\end{lem}


\emph{Prueba.} \\ \\


Del \textbf{lema 4} tenemos que $G$ contiene elementos parab\'olicos
y con punto en com\'un $u$ como centro. \\ \\

Consideremos un Orociclo $K$ con punto en el infinito $u$, sobre $K$
el elemento parab\'olico tiene un desplazamiento fijo . \\ \\


Sea $S_{u} < G$ el subgrupo de $G$ que consta de elementos
parab\'olicos con punto en el infinito $u$ y sea $x \in K$, la clase
de equivalencia $S_{u}x$ es un subconjunto de $K$ y podemos
distinguir dos casos.

\begin{enumerate}
\item Los puntos forman una sucesi\'on de puntos equidistantes
cuando $S_{u}$ es dinscontinuo.

\item El conjunto de puntos es denso en todo $K$ cuando $S_{u}$ no
es discontinuo.
\end{enumerate}

Para probar (1), sea $d=min \lbrace d_{H}(x,fx) | f \in S_{u}
\rbrace$ este minimo existe por que si no es asi se tendria un punto
de acumulaci\'on lo cual por hip\'otesis no es posible. Si $y \in
S_{u}x $ es claro que $min \lbrace d_{H}(y,fy) | f \in S_{u} \rbrace
= d$ y de esto concluimos que el conjunto de puntos es equidistante
de distancia $d$. \\ \\

Para el caso (2) por hip\'otesis tenemos que $S_{u}x$ se acumula en
$x$ y queremos ver que se acumula en todo $K$. Sea $k \in K$ un
punto arbitrario y $B_{\eps}(k)= \lbrace z \in D| d_{H}(k,z) < \eps
\rbrace$ y tomemos $B_{\eps}(k) \cap K$, queremos ver que esxiste $x
\in S_{u}x$ tal que $d_{H}(k,x) < \eps$ , si dicha $x$ no existe
entonces los desplazamientos  en $K$ de los elementos de $S_{u}$
estarian acotados inferiormente por $\eps$ lo cual implicaria que
$S_{u}$ es discontinuo lo que contradice la hip\'otesis, por lo
tanto $S_{u}$ en denso en todo $K$. \\ \\

Ahora supongamos que estamos en el caso (1) y sea $g_{0}$ el
elemento con el menos desplazamiento respecto de $K$
(\textcolor[rgb]{1.00,0.00,0.00}{dibujo}) y $f$ un elemento
hiperb\'olico en $G$ tal que su eje $A_{f}$ tiene a $u$ como punto
fijo y cuyo sentido es tal que se aleja de $u$
(\textcolor[rgb]{1.00,0.00,0.00}{existe?}), entonces $fg_{0}f^{-1}$
fija $u$ y es un elemento parab\'olico y por tanto $fg_{0}f^{-1} \in
S_{u}$. \\ \\

La imagen de $A_{f}$ bajo $fg_{0}f^{-1}$ es claramente la misma que
la de su imagen bajo $fg_{0}$, la recta $fg_{0}A^{f}$ se encuentra
entre $A_{f}$ y $g_{0}A_{f}$ (\textcolor[rgb]{1.00,0.00,0.00}{Por
que?-figura hiperciclo}), luego el desplazamiento de $fg_{0}f^{-1}$
es menor que el de $g_{0}$ lo cual es una contradicci\'on, por lo
tanto $S_{u}x$ es denso en $K$, es decir los desplazamientos de los
elementos de $S_{u}$ son tan peque�os como se quiera. \\ \\


Como $G$ no tiene a $u$ como punto invariante, sea $g \in G$ tal que
$gu \neq u$. Sean $f_{1},f_{2}$ elementos hiperb\'olicos
pertenecientes a los dos ejes del lema, al menos uno de los
elementos hiperb\'olicos $gf_{j}g^{-1}$ no deja invariante $u$, ya
que $g$ solo puede mover $u$ a los m\'as a uno de los puntos
extremos de $A_{f_{j}}$. Entonces existe en $G$ un elemento
hiperb\'olico $f$ que no deja $u$ invariante. Elijamos ahora un
orociclo $K$ tal que intersecte $A_{f}$
(\textcolor[rgb]{1.00,0.00,0.00}{dibujo}) y elejimos $h$ tal que su
desplazamiento respecto a $K$ ayude a que los ejes $A_{f}$ y
$A_{hf}$ de $f$ y $f'=hfh^{-1}$ respectivamente se intersecten en un
\'angulo arbitrariamente peque�o
(\textcolor[rgb]{1.00,0.00,0.00}{?}), e particular elegimos $h$ tal
que
$$sen\phi senh^{2} \frac{1}{2} \lambda_{f} < 1$$

ya que $senh^{2} \frac{1}{2}\lambda_{f}$ es fijo y $sen\phi
\rightarrow 0$ cuando $\phi \rightarrow 0$, y del \textbf{lema 2} se
sigue que $[f.f']$ es el\'iptico. $_{\square}$ \\ \\


\section{6}

Ahora sea $G < PSL(2,\mathbb{R})$ sin puntos invariantes y sin
elementos el\'ipticos. \\ \\

Antes que nada veamos que $G$ no es abeliano. Si los ejes son
divergentes y  $fg = gf$ entonces $fgf^{-1} = g$, pero no es posible
ya que $fgf^{1}$ fija $fx,fy$ con $x,y $ puntos fijos de $g$, pero
esto implica $fx=y$ y $fy=x$ pero esto implica $f^{2}x=x$, pero
$f^{2}$ fija los mimos puntos que $f$ por tanto no puede fijar los
puntos de $g$. (\textcolor[rgb]{1.00,0.00,0.00}{dibujo}). \\ \\

Es analogo si los ejes son paralelos ya que $f$ no puede
intercambiar $x$ y $y$ al mismo tiempo.
(\textcolor[rgb]{1.00,0.00,0.00}{dibujo}). \\ \\

(\textcolor[rgb]{1.00,0.00,0.00}{considerar el caso de una
parab\'olico e hiperbolico y 2 parabolicos}) \\ \\


M\'as a\'un debe haber elementos hiperb\'olicos en $G$, de otro modo
solo tendr\'ia elementos parab\'olicos que por hip\'otesis no
tendrias un punto fijo en com\'un y el \textbf{lema 3} implicaria
una contradicci\'on en este caso. \\ \\

Sea $f$ hiperb\'olico en $G$ y $A_{f} $ su eje y $\lambda_{f}$ la
medida de su desplazamiento, sea $g \in G$ tal que $fg \neq gf$,
entonces $f' = gf^{\pm 1}g^{-1}$ es una traslaci\'on con el mismo
desplazamiento que $f$ y con eje $gA_{f} \neq A_{f}$ (Por no ser
permutables) [Si $A_{f}=gA_{f} \Rightarrow gfg^{-1}=f$ o
$gfg^{-1}=f^{-1}$ el caso 1 no es posible, checar el caso 2]. \\ \\

$A_{f},A_{f'}$ no pueden ser paralelas por el \textbf{lema 5}, si
son divergente sea $\delta$ la distancia entre ellas y elijamos $f'$
tal que el exponente de $f$ haga $f$ y $f'$ con desplazamientos
opuestos, por las condiciones impuestas en $G$, $ff'$ no es
el\'iptico y del \textbf{lema 1} obtenemos

$$ (1) \  senh \frac{1}{2}
\delta senh \frac{1}{2} \lambda_{f} \geqq 1$$

Si $A_{f},A_{f'}$ son concurrentes sea $\phi$ su \'angulo de
intersecci\'on, dado que $[f,f']$ no es una rotaci\'on,  del
\textbf{lema 2} tenemos

$$(2) \ sen \phi senh^{2}\frac{1}{2} \lambda_{f} \geqq 1 $$

Estas desigualdades se aplican a $g$ fijo y $f$ variando
\textcolor[rgb]{1.00,0.00,0.00}{(que cumplan la condici\'on)}

(\textcolor[rgb]{1.00,0.00,0.00}{dibujo}) \\ \\



Sea $A$ un eje de un elemento hiperb\'olico en $G$, los elementos
hiperb\'olicos que comparten este eje forman un subgrupo abeliano
$S_{A}$ de $G$, sea $g \in G$ un elemento que no tenga como eje a
$A$, es decir que no permuta con los elementos de $S_{A}$, se tiene
que $gA \neq A$ y cuya distancia a $A$ digamos $d_{H}(gA,A)= \delta$
o corta a $A$ en un \'angulo $\phi$ dependiendo si $A_{g}$ y $A$ son
divergentes o concurrentes, dependiendo cual sea el caso aplicaremos
$(1)$ o $(2)$. \\ \\


Si $f$ varia sobre todos los elementos de $S_{A}$, podemos notar que
todos tienen desplazamiento mayor que un n\'umero positivo distinto
de 1 (\textcolor[rgb]{1.00,0.00,0.00}{�por que?}) (por las
condiciones 1 y 2 la cota inferior no puede ser 1) \\

Entonces $S_{A}$ es discontinuo. Sea $\lambda_{0}$ el desplazamiento
mas peque�o de elementos de $S_{A}$ (si existe). \\

Podemos aplicar las mismas desigualdades para $f \in S_{A}$ fijo y
$g $ variando sobre los representantes $g_{1},g_{2},...$ de las
clases de $S_{A} < G$ ($g_{1} \backsim g_{2} \Leftrightarrow
g_{1}S_{A} = g_{2}S_{A} \Leftrightarrow g_{1}g_{2}^{-1} \in S_{A}$).

\begin{rem}
Si $g_{1} \backsim g_{2} \Rightarrow g_{1}A=g_{2}A$ ya que
$g_{1}g_{2}^{-1} = f \in S_{A} \Rightarrow g_{1}g_{2}^{-1}A = fA
\Rightarrow g_{1}g_{2}^{-1}A = A \Rightarrow g_{2}^{-1}A =
g_{1}^{-1}A$
\end{rem}

Entonces los ejes de los elementos hiperb\'olicos $f_{v}=
g_{v}fg_{v}^{-1}$ varian sobre los diferentes ejes de la clase de
equivalencia $GA$ de $A$. \\ \\

Dado que $f_{v}$ tiene el mismo desplazamiento $\lambda_{f}$ que $f$
para todo \'indice $v$, de $(1)$ y $(2)$ obtenemos que
$senh\frac{1}{2} \delta $ y $sen \phi$ tienen una cota inferior por
lo que $\delta$ y $\phi$ tambi\'en la tienen seg\'un sean
divergentes o concurrentes los ejes $g_{v}A$, esto implica que los
ejes no se acumulan en $D$ (\textcolor[rgb]{1.00,0.00,0.00}{dibujo})
\\ \\


Sea $x \in A$. Todo punto en $Gx$ esta en alg\'un elemento del
conjunto $g_{v}A$. Para un eje dado los puntos $Gx$ que se
encuentran en este eje forman una sucesi\'on de puntos equidistantes
a distancia $\lambda$ [Los puntos en $Gx \cap g_{v}A$ son los puntos
$gx,hx$ tal que $g \backsim h $, luego para $gx,hx \in g_{v}A
\Rightarrow d_{H}(gx,hx)=d_{H}(x,g^{-1}hx), g^{-1}h \in S_{A}
\Rightarrow \lambda_{g^{-1}h} \geqq \lambda_{0}$]
(\textcolor[rgb]{1.00,0.00,0.00}{dibujo}). \\

Luego un c\'irculo con centro $x$ y radio $\frac{1}{2} \lambda_{0}$
no puede contener mas de un punto de dicha sucesi\'on en su
interior, entonces el n\'umero de puntos en $Gx$ dentro de este
c\'irculo es a lo m'as igual al n\'umero de ejes $g_{v}A$ que cortan
al c\'irculo, este n\'umero es finito ya que los ejes $g_{v}A$ no se
acumulan en $D$, $\therefore$ $x$ no es punto de acumulaci\'on de
$Gx$. $_{\square}$





























































































% ----------------------------------------------------------------
\bibliographystyle{amsplain}
\bibliography{a}
\end{document}
% ----------------------------------------------------------------
