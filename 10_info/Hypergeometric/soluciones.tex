% ----------------------------------------------------------------
% Article Class (This is a LaTeX2e document)  ********************
% ----------------------------------------------------------------
\documentclass[12pt]{article}
\usepackage[english]{babel}
\usepackage{amsmath,amsthm}
\usepackage{amsfonts}
% THEOREMS -------------------------------------------------------
\newtheorem{thm}{Theorem}[section]
\newtheorem{cor}[thm]{Corollary}
\newtheorem{lem}[thm]{Lemma}
\newtheorem{prop}[thm]{Proposition}
\theoremstyle{definition}
\newtheorem{defn}[thm]{Definition}
\theoremstyle{remark}
\newtheorem{rem}[thm]{Remark}
\numberwithin{equation}{section}
% ----------------------------------------------------------------
\begin{document}

\title[1]{2}%
\author{San}%

% ----------------------------------------------------------------
\begin{abstract}
  qq
\end{abstract}
\maketitle
% ----------------------------------------------------------------
\section{Soluciones de la ecuaci\'on hipergeom\'etrica }

Consideremos la siguiente serie :

$$F(a,b,c;x)= \sum_{n=0}^{\infty } \frac{(a)_{n} (b)_{n}}{(c)_{n}(1)_{n}} x^{n}$$

En donde $x$ es una variable compleja , $(\alpha)_{n} = \alpha (\alpha +1)(\alpha +2)\cdots (\alpha +n-1)$, %cuando n= 0? %

y $c \neq 0, -1, -2,...$. Esta serie es la llamada serie hipergeometrica.  El coeficiente $n$-esimo de esta serie al que llamaremos $A_{n}$ esta dado por $$A_{n}=\frac{a(a+1)\cdots (a+n-1) \cdot b(b+1)\cdots(b +n -1) }{c(c+1)\cdots (c+n-1) \cdot 1 (1+1) \cdots (n)}$$ De esta igualdad obtenemos:

$$\frac{A_{n+1}}{A_{n}}= \frac{(a+n)(b+n)}{(c+n)(1+n)}$$

Usando el criterio %del cociente%
Podemos notar que el radio de convergencia de esta serie es 1, excepto en el caso de que $a$ o $b$ sean negativos ya que en cualquiera de estos casos la serie es finita. De cualquier modo tenemos que dicha serie define una funci\'on holomorfa en $x$ en al menos el disco de radio 1 centrado en 0 y tambi\'en es holomorfa en $(a,b,c)$ si $c \neq 0, -1, -2,...$ \\

Nuestro interes por la serie definida anteriormente viene del hecho que dicha serie es una soluci\'on de la ecuaci\'on diferencial hipergeometrica

$$ x(1-x)\frac{d^{2}u}{dx^{2}} + \lbrace c- (a+b+1)x \rbrace \frac{du}{dx} -abu=0 $$

 A la cual denotamos $E(a,b,c;x)$ y la referimos como la serie hipergeom\'etrica con parametros $(a,b,c)$. Para ver que $F(a,b,c;x)$ es una soluci\'on de $E(a,b,c)$ procedemos como Euler e introducimos el operador $ D = x \frac{d}{dx}$, a continuaci\'on  enunciamos algunas  propiedades que ser\'an \'utiles:

\begin{enumerate}
\item $Dx^{n}= nx^{n}$
\item $f(D)x^{n}=f(n)x^{n}$ donde $f$ es un polinomio de coeficientes constantes
\end{enumerate}

La propiedad n\'umero 1 es clara de la definici\'on, para la propiedad n\'umero 2 debemos dejar en claro como funciona el operador $D^{n}$ al aplicarlo a $x^{n}$. Cuando $n=2$ tenemos que $D^{2} x^{n} = x\frac{d}{dx} (x\frac{d}{dx} (x^{n}) )= x\frac{d}{dx}(x \cdot nx^{n-1})=x\frac{d}{dx} (nx^{n} ) = x\cdot n^{2}x^{n-1}=n^{2}x^{n}$, en general se tiene por un proceso an\'alogo al anterior que $D^{m}x^{n}=n^{m}x^{n}$.Sea

$$ f(x)= \sum_{i=0}^{n} a_{i}x^{i} $$

Donde $a_{n} \neq 0$, un polinomio de grado n y por $f(D)$ entendemos el operador $$f(D) = \sum_{i=0}^{n} a_{i}D^{i}$$

Entendiendo $D^{i}$ como la composici\'on del operador $D$ $i$-veces consigo mismo. Aplicamos este operador a $x^{n}$ y obtenemos

$$f(D)x^{n}=(\sum_{i=0}^{n}a_{i}D^{i}) x^{n} = \sum_{i=0}^{n}a_{i} D^{i}( x^{n}) = \sum_{i=0}^{n}a_{i}n^{i}x^{n} = f(n)x^{n}   $$


Gracias a este operador $D$ podemos escribir la ecuaci\'on diferencial hipergeom\'etrica en t\'erminos de dicho operador para obtener;

$$E(a,b,c): [(a+D)(b+D)- (c+D)(1+D)\frac{1}{x}]u=0$$

Con esta nueva presentaci\'on verificamos que la serie $F(a,b,c;x)$ es una soluci\'on de dicha ecuaci\'on. \\

Podemos escribir 

$$F(a,b,c;x)= \sum_{n=0}^{\infty } \frac{(a)_{n} (b)_{n}}{(c)_{n}(1)_{n}} x^{n} = \sum A_{n} x^{n}$$

Y evaluamos

$$[(a+D)(b+D)- (c+D)(1+D)\frac{1}{x}] \sum A_{n}x^{n}$$
$$= \sum [(a+D)(b+D)A_{n}x^{n}- (c+D)(1+D)A_{n}x^{n-1}]$$
$$= \sum [(a+n)(b+n)A_{n}x^{n}- (c+n-1)(1+n-1)A_{n}x^{n-1}]$$
$$= \sum [(a+n)(b+n)A_{n}x^{n}- (c+n)(1+n)A_{n}x^{n}]=0$$


La \'ultima igualdad se da gracias a que obtenemos una serie telesc\'opica. Esta serie tiene singularidades en $0,1$ e $\infty$, nuestro objetivo ahora es encontrar soluciones en los puntos singulares como hicimos anteriormente con $x=0$. Denotaremos  a la soluci\'on  $F(a,b,c;x)$ como $f_{0}(x;0)$. \\

Queremos hallar otra soluci\'on en $x=0$ pero para esto necesitamos calcular la aplicaci\'on del operador $D$ al t\'ermino $x^{s} u$;

$$D(x^{s}u)= sx^{s}u + x^{s}Du = x^{s}(s +D)u$$

esto es $Dx^{s}= x^{s} (s + D)$, con este c\'alculo podemos obtener una expresi\'on equivalente para 

$$[(a+D)(b+D) - (c+d)(1+d) \frac{1}{x}] x^{1-c} $$

La expresi\'on anterior es equivalente a $x^{1-c}[(a+1-c+D)(b+1-c+D)-(1+D)(2-c+D)\frac{1}{x}]$ y esto nos dice que 

$$x^{1-c} F(a+1-c,b+1-c,2-c:x) $$

Es tambi\'en una soluci\'on de $E(a,b,c)$, aunque la serie anterior esta definida solamente cuando $2-c$ no es un entero negativo o cero (cuando es $c=1$ la serie  coincide con $F(a,b,c;x)$) esto no afecta nuestra soluci\'on ya que tomamos $1-c \in i\mathbb{R}$ en cuyo caso siempre esta definida. A esta ultima soluci\'on la denotamos por $f_{0}(x;1-c)$.

Queremos hallar tambi\'en soluciones alrededor del punto singular $x=1$, llevamos a cabo una transformaci\'on de la variable $x$ en $1-x$ abusando de la notaci\'on, y verificamos como luce ahora $E(a,b,c)=x(1-x) \frac{d^{2}u}{dx^{2}} + \lbrace c- (a+b+1)x \rbrace \frac{du}{dx} -abu$. Bajo esta transformaci\'on el primero y el tercer t\'ermino no cambian, pero el segundo t\'ermino ahora es 

$$-\lbrace  c- (a+b+1)(1-x)\rbrace = a+b+1-c-(a+b+1)x $$

Entonces podemos notar que la ecuaci\'on transformada es de nuevo la ecuaci\'on hipergeom\'etrica con parametros $ (a,b,a+b+1-c)$ y a menos que $c-a-b$ sea entero obtenemos como antes dos soluciones pero ahora alrededor de $x=1$

$$F(a,b,a+b+1-c;1-x), \ \ (1-x)^{c-a-b}F(c-a,c-b,c+1-a-b;1-x)$$

En nuestro caso tomamos $c-a-b \in i\mathbb{R} $ por lo que las soluciones anteriores siempre estan definidas en su radio de convergencia. A las soluciones anteriores las denotamos $f_{1}(x;0) $ y $f_{1}(x;c-a-b)$ respectivamente.

\end{document}
% ----------------------------------------------------------------
